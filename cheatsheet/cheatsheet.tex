\documentclass[10pt,landscape,letterpaper]{article}
\usepackage{multicol}
\usepackage{calc}
\usepackage{ifthen}
\usepackage[landscape,letterpaper]{geometry}
\usepackage{amsmath,amsthm,amsfonts,amssymb}
\usepackage{color,graphicx,overpic}
\usepackage{hyperref}
\usepackage[version=4]{mhchem}
\usepackage{framed}

\pdfinfo{
  /Title (che102 cheat sheet)
  /Creator (TeX)
  /Producer (pdfTeX 1.40.0)
  /Author (Ricky Tang)
  /Subject (Chemistry)
  /Keywords (chemistry, che102, cheatsheet)}

% Super narrow margins
\ifthenelse{\lengthtest { \paperwidth = 11in}}
    { \geometry{top=.3in,left=.3in,right=.3in,bottom=.3in} }
    {\ifthenelse{ \lengthtest{ \paperwidth = 297mm}}
        {\geometry{top=0.5cm,left=0.5cm,right=0.5cm,bottom=0.5cm} }
        {\geometry{top=0.5cm,left=0.5cm,right=0.5cm,bottom=0.5cm} }
    }

% Turn off header and footer
\pagestyle{empty}

% Redefine section commands to use less space
\makeatletter
\renewcommand{\section}{\@startsection{section}{1}{0mm}%
                                {-1.5ex plus -.5ex minus -.2ex}%
                                {1.5ex plus .3ex}%
                                {\normalfont\Large\bfseries\underline}}
\renewcommand{\subsection}{\@startsection{subsection}{2}{0mm}%
                                {-1.2ex plus -.5ex minus -.2ex}%
                                {1.2ex plus .3ex}%
                                {\normalfont\large\bfseries}}
\renewcommand{\subsubsection}{\@startsection{subsubsection}{3}{0mm}%
                                {-1ex plus -.4ex minus -.15ex}%
                                {1ex plus .2ex}%
                                {\normalfont\normalsize\bfseries}}
\makeatother

% Define BibTeX command
\def\BibTeX{{\rm B\kern-.05em{\sc i\kern-.025em b}\kern-.08em
    T\kern-.1667em\lower.7ex\hbox{E}\kern-.125emX}}

% Don't print section numbers
\setcounter{secnumdepth}{0}

\setlength{\parindent}{0pt}
\setlength{\parskip}{1.5pt plus 0.5ex}
\makeatother

% Reduce line spacing
\linespread{0.85}

%My Environments
\newtheorem{example}[section]{Example}
% -----------------------------------------------------------------------

\begin{document}
\raggedright
\footnotesize
\begin{multicols}{4}

  % multicol parameters
  % These lengths are set only within the two main columns
  %\setlength{\columnseprule}{0.25pt}
  \setlength{\premulticols}{1pt}
  \setlength{\postmulticols}{1pt}
  \setlength{\multicolsep}{1pt}
  \setlength{\columnsep}{2pt}

  \section{pre-midterm stuff}

  \section{rates of reaction}

  \subsection{collision theory}

  \textbf{conditions for rxn}: collide + sufficient $E_a$ + correct orientation

  \textbf{catalyst:}
  \begin{itemize}
    \item increases rate via \textbf{lower $E_a$ path}
    \item participates but \textbf{not altered} by rxn
    \item does \textbf{not appear} in overall rxn
    \item does \textbf{NOT} change equilibrium
  \end{itemize}

  \textbf{factors affecting rxn rate:}
  \begin{itemize}
    \item concentration of reactants
    \item temperature
    \item presence of \textbf{catalysts}
    \item physical nature of reactants
  \end{itemize}

  \textbf{$E_p$ diagrams}: enthalpy of reactants and products are \textbf{FIXED} regardless of catalyst presence.

  \subsection{reaction rates}

  must be determined \textbf{experimentally}

  rxn rate can get the rate of change of single species using \textbf{mole ratio}

  $\displaystyle \dot{R}=-\frac{\dot{R_{A}}}{a}\implies \dot{R}_{A}=-a\times \dot{R}$

  unit: $M/s$ ($\frac{mol}{L \cdot s}$)

  \textbf{positive 4 products, negative 4 reactants}

  \subsection{rate laws}

  relates rate to \textbf{concentration} of reactants

  \subsubsection{differential}

  for $aA+bB \rightarrow \text{products}$:
  $\dot{R}=-\frac{1}{a}\frac{d[A]}{dt}=-\frac{1}{b}\frac{d[B]}{dt}=k[A]^{m}[B]^{n}$

  \begin{itemize}
    \item $m$: order w.r.t. A  ;  $n$: order w.r.t. B
    \item $m+n$: overall order
    \item $k$: rate constant (only changes with \textbf{T})
  \end{itemize}

  \textbf{note:} exponents $\neq$ coefficients

  \subsubsection{integrated \& half-life}

  for $aA \rightarrow \text{products}$:

  \textbf{zero-order:}
  \begin{itemize}
    \item $[A]=-akt+[A]_{0}$
    \item unit of $k$: $M/s$
    \item $t_{1/2}=\frac{[A]_{0}}{2ak}$
  \end{itemize}

  \textbf{first-order:}
  \begin{itemize}
    \item $\ln[A]=-akt+\ln[A]_{0}$
    \item unit of $k$: $1/s$
    \item $t_{1/2}=\frac{\ln~2}{ak}$
  \end{itemize}

  \textbf{second-order:}
  \begin{itemize}
    \item $\frac{1}{[A]}=akt+\frac{1}{[A]_{0}}$
    \item unit of $k$: $1/(M\cdot s)$
    \item $t_{1/2}=\frac{1}{ak[A]_{0}}$
  \end{itemize}

  \textbf{n-order ($n \neq 1$):}
  \begin{itemize}
    \item (see formula sheet)
    \item unit of $k$: $\frac{1}{\text{M}^{n-1} \cdot \text{s}}$
    \item $t_{1/2}=\frac{2^{n-1} - 1}{ak(n-1)[A]_0^{n-1}}$
  \end{itemize}

  \subsection{arrhenius equation}

  preferrably use two-point form
  
  use $R = 8.314$ J/(mol $\cdot$ K)

  \vfill\null
  \columnbreak

  \section{phase equilibrium}

  \subsection{phase equilibrium}

  \textbf{defn:} rate (forward) = rate (reverse)
  
  e.g. vapour liquid equilibrium $\implies$ $\text{rate}_{\text{vap}} = \text{rate}_{\text{cond}}$

  \textbf{note:} phase $\neq$ state of matter

  \textbf{phase changes:}
  $\text{gas} \xrightarrow{\text{condensation}} \text{liquid} \xrightarrow{\text{solidification}} \text{solid}$
  $\text{gas} \xleftarrow{\text{vaporization}} \text{liquid} \xleftarrow{\text{fusion}} \text{solid}$
  $\text{gas} \xleftarrow[\text{deposition}]{\text{sublimation}} \text{solid}$

  $\Delta H_{\text{fus}} + \Delta H_{\text{vap}} = \Delta H_{\text{sub}}$

  \subsection{vapour pressure}

  \textbf{vapour pressure:} pressure of vapour at equilibrium
  
  weak IMF $\rightarrow$ lower $T_{\text{bp}}$ $\rightarrow$ higher vapour pressure (more volatile); and \textbf{vice versa}

  $$P_{\text{vap}} = f(T, \text{type of liquid})$$

  \textbf{evaporation vs. boiling:}
  \begin{itemize}
    \item \textbf{evaporation:} at surface, any temperature
    \item \textbf{boiling:} throughout liquid, specific temperature ($T_{\text{bp}}$)
  \end{itemize}

  \textbf{boiling point:} temperature where $P_{\text{vap}}(T_{\text{bp}}) = P$. 
  
  normal boiling point: $P_{\text{vap}} = 1$ atm

  \subsubsection{clausius-clapeyron equation}

  (see formula sheet)

  relates vapour pressure to \textbf{temperature}

  can be used for \textbf{ANY phase change} as long as the right enthalpy is used

  \subsubsection{non-equilibrium stuff (humidity/saturation)}

  $\displaystyle\%\text{ saturation} = \frac{P_{A}}{P_{A}^{\text{vap}}(T)} \times 100\%$

  \textbf{dew point $T_{\text{dp}}$:} temperature where humid air reaches saturation

  condensation occurs at \textbf{100 \% saturation}

  humidity refers specifically to $\ce{H2O}$

  \subsection{phase diagrams}

  can be used to get the vapour pressure at a temperature (using the equilibrium lines)

  \textbf{triple point:} solid, liquid, gas coexist $\rightarrow$ there can be multiple triple points, but only one is solid-liquid-gas equilibrium

  \textbf{equilibrium line:} 2 phases in equilibrium

  \textbf{critical point:} point where substance becomes supercritical fluid

  \textbf{supercritical fluid:} neither liquid nor gas, but having properties of both

  \textbf{polymorphism:} existence of solid in more than one form

  \textbf{note:} a substance can phase change at multiple different temperatures or pressures, but at a given P (T), it phase changes at the corresponding T (P).

  \subsection{henry's law}

  \textbf{essentially:} gas \textbf{solubility} in liquid increases with increasing \textbf{pressure}

  \textbf{ideal solution:} $\Delta H_{\text{soln}} = 0$, similar forces between all components

  use $P_A = H_A x_A$ when constants have units of \textbf{pressure}, and working with \textbf{mole fractions}

  use $C_A = H_A x_A$ when constants have units of \textbf{pressure \& concentration}, and working with \textbf{concentrations}

  \subsection{raoult's law}

  \textbf{essentially:} adding solute \textbf{lowers} the vapour pressure of solvent

  $P_A = x_A P_A^{\text{vap}}$ $\rightarrow$ use this if u are given vapour pressures

  applies to ideal solutions or dilute solutions ($x_{\text{solv}} > 0.98$)

  solute and solvent both in vapour and in solution

  \textbf{deviations:} positive if total pressure is greater than each individual pure vapour pressure (and vice versa)

  \subsection{colligative properties}

  \textbf{defn:} properties of solutions that depend on the \textbf{RATIO} of solute particles to solvent molecules (NOT type of solute)

  \subsubsection{vapour pressure lowering}

  we can use vapour pressure to estimate the \textbf{molar mass} of an unknown solid dissolved in a known liquid

  $$M_{\text{solid}}=-\frac{M_{\text{liquid}}m_{\text{solid}}}{m_{\text{liquid}}}\left( \frac{P^{\text{vap}}_{\text{liquid}}}{\Delta P_{\text{liquid}}}+1 \right)$$

  \subsubsection{b.p. elevation \& f.p. depression}

  $\Delta T_{\text{bp}} = i K_b m$, $\Delta T_{\text{fp}} = -i K_f m$

  $m$ is the solute \textbf{MOLALITY (moles solute/kg solvent)}!

  $K_b$ and $K_f$ are constants, dependent on the \textbf{solvent} only

  \textbf{van't hoff factor $i$} $\rightarrow$ for ionic compounds 
  
  $i_{\text{min}} = 1$ (pure solid/liquid, no dissociation)

  $i_{\text{max}} = \text{\# of ions}$ (complete dissociation), e.g. 2 for \ce{NaCl}

  $\displaystyle\text{\% dissociation} = \frac{i - 1}{i_{\text{max}} - 1} \times 100\%$

  \columnbreak

  \section{chemical equilibrium}

  \subsection{equilibrium constants}

  \textbf{chemical equilibrium:} at equilibrium, forward rate = reverse rate

  for $aA + bB \rightleftharpoons cC + dD$:

  \textbf{equilibrium constant $K_c$:}
  $$K_c = \frac{[C]^c[D]^d}{[A]^a[B]^b}$$

  where [ ] = equilibrium concentration in mol/L. $K_c$ is constant at given temperature, independent of initial concentrations

  \textbf{equilibrium constant $K_P$:}
  $$K_P = \frac{P_C^c P_D^d}{P_A^a P_B^b}$$

  partial pressures at equilibrium, use units of bar. $K_P$ is constant at given temperature

  \textbf{relationship:}
  $$K_P = K_c(RT)^{\Delta n} \quad \text{where} \quad \Delta n = (c + d) - (a + b)$$

  $R = 0.08314$ L$\cdot$bar/(mol$\cdot$K)

  \textbf{heterogeneous reactions:} pure liquids and solids are \textbf{not included} in equilibrium constant expressions

  \textbf{magnitude of $K$:}
  \begin{itemize}
    \item $K > 10^{10}$: reaction goes to completion
    \item $K < 10^{-10}$: reaction does not occur forward
    \item $10^{-10} < K < 10^{10}$: equilibrium calculation needed
  \end{itemize}

  \subsection{properties of equilibrium constants}

  \begin{itemize}
    \item multiply reaction by constant $\rightarrow$ raise $K$ to that power
    \item reverse reaction: $K_{\text{reverse}} = \frac{1}{K_{\text{forward}}}$
    \item add reactions: multiply equilibrium constants
  \end{itemize}

  \subsection{reaction quotient \& direction}

  \textbf{reaction quotient $Q$:}
  $$Q_C = \frac{[C]^c[D]^d}{[A]^a[B]^b} \quad Q_P = \frac{P_C^c P_D^d}{P_A^a P_B^b}$$

  calculated same as $K$ but system doesn't need to be at equilibrium

  \textbf{direction of change:}
  \begin{itemize}
    \item $Q_c < K_c$: reactants in excess, forward reaction
    \item $Q_c = K_c$: at equilibrium
    \item $Q_c > K_c$: products in excess, reverse reaction
  \end{itemize}

  \subsection{le chatelier's principle}

  system responds to change by attaining new equilibrium that partially offsets the change

  \textbf{change in concentration:}
  \begin{itemize}
    \item add reactant: $Q_c < K_c$, forward reaction, more products
    \item add product: $Q_c > K_c$, reverse reaction, more reactants
  \end{itemize}

  \textbf{change in volume:} reduce volume $\rightarrow$ shifts towards side with fewer gas molecules. effect negligible for condensed phases

  \textbf{change in pressure:}
  \begin{itemize}
    \item add/remove gas: affects equilibrium
    \item change volume: affects equilibrium if gas present
    \item add inert gas at constant volume: no effect (partial pressures unchanged)
  \end{itemize}

  \textbf{change in temperature:}
  \begin{itemize}
    \item endothermic ($\Delta H > 0$): $T \uparrow \rightarrow K \uparrow \rightarrow$ shifts to products
    \item exothermic ($\Delta H < 0$): $T \uparrow \rightarrow K \downarrow \rightarrow$ shifts to reactants
    \item reaction shifts away from "energy term" as temperature increases
  \end{itemize}

  \textbf{van't hoff equation:}
  $$\ln\frac{K_2}{K_1} = -\frac{\Delta H}{R}\left(\frac{1}{T_2} - \frac{1}{T_1}\right)$$

  estimate $K$ at different temperature. clausius-clapeyron is special case where $K$ = vapour pressure

  \vfill\null
  \columnbreak

  \section{electrochemistry}

  \subsection{oxidation states \& redox reactions}

  \textbf{oxidation state (OS):} charge atom would have if all bonds were ionic. not measurable, conceptual tool

  \textbf{rules for assigning OS:}
  \begin{itemize}
    \item free element: OS = 0
    \item monatomic ion: OS = charge
    \item sum of OS in neutral species = 0
    \item sum of OS in ion = charge
    \item H, group 1: +1; group 2: +2; halogens: -1; group 16: -2; group 15: -3
  \end{itemize}

  \textbf{exceptions:} H is -1 in hydrides (e.g., \ce{LiH}); O is -1 in peroxides (e.g., \ce{H2O2})

  \textbf{redox reactions:}
  \begin{itemize}
    \item reduction: OS decreases, gains electrons. oxidizing agent causes oxidation
    \item oxidation: OS increases, loses electrons. reducing agent causes reduction
  \end{itemize}

  \textbf{balancing redox:}
  \begin{enumerate}
    \item write separate half-reactions
    \item balance atoms except H, O
    \item balance O with \ce{H2O}
    \item balance H with \ce{H+}
    \item balance charges with $e^-$
    \item add half-reactions to cancel $e^-$
    \item for basic: add \ce{OH-} equal to \ce{H+} to each side, simplify
  \end{enumerate}

  \subsection{galvanic cell \& faraday's law}

  \textbf{galvanic cell:} derives electrical energy from spontaneous redox reactions

  notation: $\text{Zn}(s)|\text{Zn}^{2+}(aq)||\text{Cu}^{2+}(aq)|\text{Cu}(s)$

  \textbf{anode (oxidation):} lower potential, electrons flow from anode. \textbf{cathode (reduction):} higher potential, electrons flow to cathode

  \textbf{current \& faraday's law:}
  $$I = \frac{Q}{t} \quad \text{units: A} = \text{C/s}$$

  $$Q = nF \quad n = \frac{Q}{F} = \frac{It}{F}$$

  where $F = 96,485$ C/mol (charge of 1 mole of electrons), $n$ = moles of electrons

  \textbf{applications:} from moles of electrons, determine moles of substance produced/consumed at electrode

  \textbf{current efficiency:}
  $$\eta = \frac{\text{charge used}}{\text{charge supplied}} \times 100\% = \frac{\text{actual mass}}{\text{theoretical mass}} \times 100\%$$

  \subsection{standard cell potential \& nernst equation}

  \textbf{cell potential $E_{\text{cell}}$:} electrical potential difference (voltage) between electrodes in V (J/C)

  \textbf{standard state:} 1.0 M for dissolved species, 1 bar for gases, 25$^\circ$C

  $$E^\circ_{\text{cell}} = E^\circ_{\text{cathode}} - E^\circ_{\text{anode}}$$

  standard: $2\text{H}^+(aq) + 2e^- \to \text{H}_2(g)$ with $E^\circ = 0$ V

  \textbf{spontaneous direction:} electrons travel from low to high potential. higher $E^\circ$ gets reduced (cathode). $E^\circ_{\text{cell}} > 0$ for spontaneous

  \textbf{nernst equation:}
  $$E_{\text{cell}} = E^\circ_{\text{cell}} - \frac{RT}{nF}\ln Q$$

  where $R = 8.314$ J/(mol$\cdot$K), $T$ = temperature (K), $F = 96,485$ C/mol, $Q$ = reaction quotient (aq: M, gas: bar), $n$ = electrons transferred

  \textbf{at 25$^\circ$C:}
  $$E_{\text{cell}} = E^\circ_{\text{cell}} - \frac{0.0592}{n}\log Q$$

  \textbf{concentration cell:} same material at anode and cathode. higher concentration acts as cathode

  \subsection{nernst at equilibrium \& electrolytic cells}

  \textbf{at equilibrium:} $E_{\text{cell}} = 0$, $Q = K$

  $$0 = E^\circ_{\text{cell}} - \frac{RT}{nF}\ln K$$

  $$K = \exp\left(\frac{E^\circ_{\text{cell}}nF}{RT}\right)$$

  use to determine equilibrium constants with galvanic cells

  \textbf{galvanic vs. electrolytic:}
  \begin{itemize}
    \item galvanic: derives electrical energy from spontaneous redox ($E_{\text{cell}} > 0$)
    \item electrolytic: uses electrical energy to promote non-spontaneous reaction ($E_{\text{cell}} < 0$)
  \end{itemize}

  \textbf{electrochemical cells:}
  \begin{itemize}
    \item anode: always oxidation. galvanic: $(-)$; electrolytic: $(+)$
    \item cathode: always reduction. galvanic: $(+)$; electrolytic: $(-)$
  \end{itemize}

\end{multicols}
\end{document}
